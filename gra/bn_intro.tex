\documentclass{article}

\usepackage[left=3cm]{geometry}
\usepackage{enumerate}
\usepackage{hyperref}
\usepackage{verbatim}
\usepackage{array}
\usepackage{longtable}
\usepackage{babel}[ngerman]
\usepackage{csquotes}[autostyle]

\hypersetup{colorlinks=false, pdfborderstyle={/S/U/W 1}}

\title{\vspace{-4cm}GRA Intro: Setup}
\date{}
\author{}

\begin{document}

\maketitle
\vspace{-1.5cm}

Da viele von euch jetzt in GRA das erste Mal mit 
einer UNIX/Linux-Umgebung arbeiten werden, hier eine kleine Einführung.\\ Ich kann und werde keinen 1:1 Tech-Support für jeden von euch leisten
können, schließlich bin ich selbst Vollzeistudent :) Bei Fragen könnt ihr aber jederzeit in den öffentlichen Zulip-Streams zu 
GRA schreiben.

\section{Betriebssystem}
    \subsection{Windows}
    Bitte installiere \href{https://learn.microsoft.com/de-de/windows/wsl/install}{Windows Subsystem for Linux (WSL)}.
    Du kannst dann jederzeit WSL einfach vom Startmenü aus starten und erhälst ein voll funktionsfähiges Ubuntu.
    Du kannst auf die in deiner WSL-Umgebung gespeicherten Daten über den Datei-Explorer zugreifen, dafür gibts den Befehl
    \verb|explorer.exe .| (den letzten Punkt nicht vergessen).\\
    Ich würde stark davon abraten, alles nativ in der Windows Command Line versuchen zum Laufen zu bringen.
    \subsection{Linux}
    Falls bereits irgendeine Linux-Distro auf deinem Rechner läuft, hast du alles
    richtig gemacht.
    \subsection{MacOS}
    Ich habe selbst nie mit MacOS gearbeitet, aber afaik sollte dieses Betriebssystem
    auf UNIX basieren, d.h. das meiste sollte ohne Probleme laufen.
\section{IDE/Editor}
\begin{itemize}
    \item Lightweight und gute Integration, aber umständliches Debugging: VSCode
    \item Sehr starke Debugging-Umgebung (ähnlich zu IntelliJ), aber manchmal ein bisschen kniffliges Setup mit Makefiles/WSL: CLion (Lizenz dafür gibt's als TUM-Studi kostenlos)
    \item Für Masochisten: VIM/Emacs/\ldots
\end{itemize}
\section{Einige nützliche Befehle}
    \begin{longtable}{p{4cm}p{12cm}}
        \verb|cd <path>| & Wechselt das current working directory zu \verb|<path>| \\ 
        \verb|ls -la| & Listet alle Dateien im CWD auf. (\verb|-l| für erweiterte Infos, \verb|-a| um auch versteckte Dateien anzuzeigen) \\  
        \verb|rm <file>|& Entfernt die Datei \verb|file|\\
        \verb|code .| & Startet VSCode im CWD\\
        \verb|make| & Startet die in der Makefile vorgebene Routine, falls sich die Ausgangsdateien geändert haben\\
        \verb|make clean| & Das selbe wie \verb|make|, nur unabhängig davon, ob sich die Ausgangsdateien geändert haben oder nicht\\
        \verb|chmod u+x <file>| & Markiert die Datei \verb|file| als ausführbar\\
        \verb|./<file>| & Führt eine Datei aus\\
        \verb|nano <file>| & Simpler Texteditor\\
        \verb|ssh <rbg-kennung>@| \verb|lxhalle.in.tum.de| & SSH-Login zur Rechnerhalle\\
        \verb|git status| & Zeigt die aktuell getrackten Dateien an\\
        \verb|git commit --allow-| \verb|empty| \verb|-m ""| & Erstellt einen leeren Commit, sodass nach einem Push auf Artemis neu gebuildet wird\\
        \verb|git diff| & Zeigt Dateiänderungen im aktuellen Git-Repo an\\
        \verb|ctrl + c| & Kein Befehl, sondern eine Tastenkombination: Unterbricht die Ausführung des aktuellen Prozesses\\
        \verb|sl| & Überlebenswichtiger Befehl\\
    \end{longtable}
\end{document}