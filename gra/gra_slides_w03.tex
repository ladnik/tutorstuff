%%
% This is an Overleaf template for presentations
% using the TUM Corporate Desing https://www.tum.de/cd
%
% For further details on how to use the template, take a look at our
% GitLab repository and browse through our test documents
% https://gitlab.lrz.de/latex4ei/tum-templates.
%
% The tumbeamer class is based on the beamer class.
% If you need further customization please consult the beamer class guide
% https://ctan.org/pkg/beamer.
% Additional class options are passed down to the base class.
%
% If you encounter any bugs or undesired behaviour, please raise an issue
% in our GitLab repository
% https://gitlab.lrz.de/latex4ei/tum-templates/issues
% and provide a description and minimal working example of your problem.
%%


\documentclass[
  german,            % define the document language (english, german)
  aspectratio=169,    % define the aspect ratio (169, 43)
  % handout=2on1,       % create handout with multiple slides (2on1, 4on1)
  % partpage=false,     % insert page at beginning of parts (true, false)
  % sectionpage=true,   % insert page at beginning of sections (true, false)
]{tumbeamer}


% load additional packages
\usepackage{booktabs}
\usepackage{graphicx}
\usepackage{tikz}
\usepackage{url}
\usepackage{pgfplots}
\usepackage{hyperref}
\usepackage{pmboxdraw}
\usepackage{float}
\usepackage{babel}[ngerman]
\usepackage{csquotes}[autostyle]
\usepackage[useregional]{datetime2}

% image path
\graphicspath{ {./resources/} }

% presentation metadata
\title{Übung 03: C Toolchain}
\subtitle{Grundlagenpraktikum Rechnerarchitektur}
\author{Niklas Ladurner}

\institute{\theChairName\\\theDepartmentName\\\theUniversityName}
\date{\DTMdisplaydate{2024}{5}{3}{-1}}

\footline{\insertauthor~|~\insertshorttitle~|~\insertshortdate}


% macro to configure the style of the presentation
\TUMbeamersetup{
  title page = TUM tower,         % style of the title page
  part page = TUM toc,            % style of part pages
  section page = TUM toc,         % style of section pages
  content page = TUM more space,  % style of normal content pages
  tower scale = 1.0,              % scaling factor of TUM tower (if used)
  headline = TUM threeliner,      % which variation of headline to use
  footline = TUM default,         % which variation of footline to use
  % configure on which pages headlines and footlines should be printed
  headline on = {title page},
  footline on = {every page, title page=false},
}

% available frame styles for title page, part page, and section page:
% TUM default, TUM tower, TUM centered,
% TUM blue default, TUM blue tower, TUM blue centered,
% TUM shaded default, TUM shaded tower, TUM shaded centered,
% TUM flags
%
% additional frame styles for part page and section page:
% TUM toc
%
% available frame styles for content pages:
% TUM default, TUM more space
%
% available headline options:
% TUM empty, TUM oneliner, TUM twoliner, TUM threeliner, TUM logothreeliner
%
% available footline options:
% TUM empty, TUM default, TUM infoline


\begin{document}

\maketitle

\begin{frame}[c]{}{}
  \begin{center}
    \LARGE  Keine Garantie für die Richtigkeit der Tutorfolien: Bei Unklarheiten/Unstimmigkeiten
    haben VL-Folien Recht!
  \end{center}
\end{frame}



\begin{frame}[c, fragile]{GNU Debugger}{}
  \begin{itemize}
    \item pwndbg -- Verbesserungen für gdb
    \item ermöglicht step-by-step debugging von Programmen
    \item Kompilieren mit \verb|-g| fügt Debugging-Symbole hinzu: \verb|gcc -g main.c|
    \item wichtige Kommandos:
    \begin{itemize}
      \item \verb|break label|: Setze breakpoint an label
      \item \verb|run/continue|: Führe Programm bis zum nächsten breakpoint aus
      \item \verb|next|: Geht über Funktionen hinweg (step over)
      \item \verb|step|: Steigt in Funktionen hinein (step into)
      \item \verb|finish|: Steigt wieder heraus
    \end{itemize}
  \end{itemize}
\end{frame}

\begin{frame}[c, fragile]{Valgrind}{}
  \begin{itemize}
    \item eignet sich zum Debugging und Profiling (Performanzmessung) von Programmen
    \item Besteht aus verschiedenen Werkzeugen
    \item hilft bspw. sehr bei schwer debuggbaren Speicherfehlern, die nur unter bestimmten 
    Umständen auftreten
  \end{itemize}
\end{frame}

\begin{frame}[c, fragile]{GNU Make}{}
  \begin{itemize}
    \item Makefiles ermöglichen einen übersichtlicheren Buildvorgang
    \item Definition verschiedener Targets (zu kompilierenden Dateien) mit komplexen Abhängigkeiten
    \item \verb|make| kompiliert Target nur neu, wenn sich Ausgangsdateien geändert haben!
  \end{itemize}
\end{frame}

\begin{frame}[c]{}{}
  \begin{center}
    \LARGE Fragen?
  \end{center}
\end{frame}

\begin{frame}[fragile, c]{Links}{}
  \begin{itemize}
    \item Zulip: \href{https://zulip.in.tum.de/#narrow/stream/2267-GRA-Tutorium---Gruppe-20}{\enquote{GRA Tutorium - Gruppe 20}}
          bzw. \href{https://zulip.in.tum.de/#narrow/stream/2269-GRA-Tutorium---Gruppe-22}{\enquote{GRA Tutorium - Gruppe 22}}
    \item \href{https://github.com/pwndbg/pwndbg}{pwndbg}
    \item \href{https://valgrind.org/docs/manual/manual.html}{Dokumentation zu Valgrind}
    \item \href{https://www.gnu.org/software/make/manual/html_node/index.html}{Dokumentation zu GNU Make}
  \end{itemize}
\end{frame}

\maketitle

\end{document}
