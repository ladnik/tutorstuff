%%
% This is an Overleaf template for presentations
% using the TUM Corporate Desing https://www.tum.de/cd
%
% For further details on how to use the template, take a look at our
% GitLab repository and browse through our test documents
% https://gitlab.lrz.de/latex4ei/tum-templates.
%
% The tumbeamer class is based on the beamer class.
% If you need further customization please consult the beamer class guide
% https://ctan.org/pkg/beamer.
% Additional class options are passed down to the base class.
%
% If you encounter any bugs or undesired behaviour, please raise an issue
% in our GitLab repository
% https://gitlab.lrz.de/latex4ei/tum-templates/issues
% and provide a description and minimal working example of your problem.
%%


\documentclass[
  german,            % define the document language (english, german)
  aspectratio=169,    % define the aspect ratio (169, 43)
  % handout=2on1,       % create handout with multiple slides (2on1, 4on1)
  % partpage=false,     % insert page at beginning of parts (true, false)
  % sectionpage=true,   % insert page at beginning of sections (true, false)
]{tumbeamer}


% load additional packages
\usepackage{booktabs}
\usepackage{graphicx}
\usepackage{tikz}
\usepackage{url}
\usepackage{pgfplots}
\usepackage{hyperref}
\usepackage{pmboxdraw}
\usepackage{float}
\usepackage{babel}[ngerman]
\usepackage{csquotes}[autostyle]
\usepackage[useregional]{datetime2}

% image path
\graphicspath{ {./resources/} }

% presentation metadata
\title{Übung 01: Zahlensysteme}
\subtitle{Einführung in die Rechnerarchitektur}
\author{Niklas Ladurner}

\institute{\theChairName\\\theDepartmentName\\\theUniversityName}
\date{\DTMdisplaydate{2023}{10}{20}{-1}}

\footline{\insertauthor~|~\insertshorttitle~|~\insertshortdate}


% macro to configure the style of the presentation
\TUMbeamersetup{
  title page = TUM tower,         % style of the title page
  part page = TUM toc,            % style of part pages
  section page = TUM toc,         % style of section pages
  content page = TUM more space,  % style of normal content pages
  tower scale = 1.0,              % scaling factor of TUM tower (if used)
  headline = TUM threeliner,      % which variation of headline to use
  footline = TUM default,         % which variation of footline to use
  % configure on which pages headlines and footlines should be printed
  headline on = {title page},
  footline on = {every page, title page=false},
}

% available frame styles for title page, part page, and section page:
% TUM default, TUM tower, TUM centered,
% TUM blue default, TUM blue tower, TUM blue centered,
% TUM shaded default, TUM shaded tower, TUM shaded centered,
% TUM flags
%
% additional frame styles for part page and section page:
% TUM toc
%
% available frame styles for content pages:
% TUM default, TUM more space
%
% available headline options:
% TUM empty, TUM oneliner, TUM twoliner, TUM threeliner, TUM logothreeliner
%
% available footline options:
% TUM empty, TUM default, TUM infoline


\begin{document}

\maketitle

\begin{frame}[c]{}{}
  \begin{center}
    \LARGE  Keine Garantie für die Richtigkeit der Tutorfolien: Bei Unklarheiten/Unstimmigkeiten 
    haben VL/ZÜ-Folien Recht!
  \end{center}
\end{frame}

\begin{frame}[c]{Organisatorisches}{}
  \begin{itemize}
    \item Wer bin ich?
    \item Wo könnt ihr mich erreichen?
    \item Mitschriften/Folien auf meiner \href{https://home.in.tum.de/~ladu/}{Homepage} (?) 
    \item Anmerkungen zu den Hausaufgaben/Übungen
    \item Tutoriumszeiten
  \end{itemize}
\end{frame}

\begin{frame}[c]{Was macht mein Rechner eigentlich?}{}
    \begin{itemize}
      \item Rechner arbeiten mit Zahlen: Bilder, Strings, ...
      \item Dezimalsystem ungeeignet: zu viele 'Zustände' (Ziffern)
      \item Strom an bzw. aus mittels Transistoren
      \item zwei Zustände $\rightarrow$ Binärsystem
      \item Oktalsystem und Hexadezimalsystem passen gut dazu!
    \end{itemize}
  \end{frame}

  \begin{frame}[c]{Wie funktioniert denn so ein Zahlensystem?}{}
      \begin{center}
        \LARGE $$W=\sum_{i=0}^na_i\cdot B^i$$
      \end{center}
      \hspace{1cm}
      \begin{itemize}
        \item Was passiert bei einem Verschieben der Stellen nach links/rechts?
        \item Wie kann man Binärzahlen in Hexadezimal/Oktal umwandeln?
      \end{itemize}
  \end{frame}

\begin{frame}[c]{Wie rechnet man im Binärsystem?}{}
  \begin{itemize}
    \item grundsätzlich gleich wie im Dezimalsystem
    \item Subtraktion: Addition mit negativer Zahl
    \item Einerkomplement vs. Zweierkomplement
  \end{itemize}
\end{frame}

\begin{frame}[c]{}{}
  \begin{center}
    \LARGE Fragen?
  \end{center}
\end{frame}

\begin{frame}[c]{Artemis-Hausaufgaben}{}
  \begin{itemize}
    \item H01-Zahlensysteme bis 29.10.2023 23:59 Uhr
    \item 100\% bedeuten 100\% - keine \emph{hidden tests} nach der Deadline
    \item kurze Einführung in Artemis/Git
  \end{itemize}
\end{frame}

\begin{frame}[fragile, c]{Links}{}
  \begin{itemize}
    \item Zulip: \href{https://zulip.in.tum.de/#narrow/stream/1917-ERA-Tutorium---Mi-1600-MI4}{\enquote{ERA Tutorium - Mi-1600-MI4}}
    bzw. \href{https://zulip.in.tum.de/#narrow/stream/1940-ERA-Tutorium---Fr-1100-MW2}{\enquote{ERA Tutorium - Fr-1100-MW2}}
    \item \href{https://www.moodle.tum.de/course/view.php?id=90679}{ERA-Moodle-Kurs}
    \item \href{https://artemis.in.tum.de/courses/288/exercises}{ERA-Artemis-Kurs}
    \item \href{https://git-scm.com/docs/gittutorial}{Git-Tutorial}, \href{https://rogerdudler.github.io/git-guide/}{alternatives Tutorial}
  \end{itemize}
\end{frame}

\maketitle

\end{document}
