\documentclass{article}

\usepackage[left=3cm]{geometry}
\usepackage{enumerate}

\title{\vspace{-4cm}Themenübersicht ERA}
\date{}
\author{}

\begin{document}

\maketitle
\vspace{-1.5cm}

\textbf{Disclaimer:} Die folgende Auflistung ist nur dazu da, einen
groben Überblick über die behandelten Themenbereiche zu geben. Die explizit
genannten Aufgaben sind Aufgaben, die ich (in eventuell leicht abgeänderter Form)
auch gut in einer Klausur sehen würde.\\
Insbesondere ist diese Auflistung weder vollständig, noch besteht die
Klausur nur aus den hier empfohlenen Aufgabentypen.\\
Es ist wie immer alles klausurrelevant, was nicht ausgeschlossen wurde
und in der Vorlesung bzw. den Übungen behandelt wurde. D.h. also auch,
dass einige Theorieinhalte und Aufgaben, die nicht in den Übungen drankamen, durchaus
in der Klausur abgefragt werden können.\\
Die mit * gekennzeichneten Aufgaben sind in ähnlicher Art schon mal in einer
Klausur drangekommen.
\begin{enumerate}
    \item Zahlensysteme
          \begin{enumerate}[i.]
              \item Binärsystem und Hexadezimalsystem\footnote{Es kommt wahrscheinlich
                        keine dedizierte Aufgabe dazu (außer vllt. Multiple Choice), aber der Umgang
                        mit verschiedenen Zahlensystemen wird voraussgesetzt, besonders bspw. bei Caches}
              \item Binärarithmetik, insbesondere Zweierkomplement
              \item Gleit- und Festkommazahlen
              \item Verständnis Overflow/Underflow
          \end{enumerate}
          Aufgaben: 1.1, 1.2
    \item RISC-V Assembly I
          \begin{enumerate}[i.]
              \item ISA
              \item RISC vs. CISC
              \item Assemblierungsprozess
              \item Befehlstypen, Register
              \item Erzeugung von Konstanten
              \item Little Endian vs. Big Endian, Speicherausrichtung
              \item grundlegende RISC-V-Befehle (RV32IM)
              \item Adressierungsarten
          \end{enumerate}
          Aufgaben: 2.2, 2.3*, 2.4, 2.5
    \item RISC-V Assembly II
          \begin{enumerate}[i.]
              \item bedingte und unbedingte Sprünge
              \item Unterprogramme
              \item Calling Convention\footnote{Ich empfehle, die RISC-V-CC gut
                        zu kennen, zumal sie nicht allzu komplex ist und leider bei den Artemis-Aufgaben
                        immer wieder verletzt wurde}
              \item Stack
              \item Von-Neumann-Architektur
          \end{enumerate}
          Aufgaben: 3.1*, 3.2*, 3.3
    \item RISC-V Assembly III
          \begin{enumerate}[i.]
              \item (End-)Rekursion
              \item Stackframes
              \item Mehrbenutzersysteme
              \item Benutzermodus und Systemmodus\footnote{Mehr dazu kommt später
              in Grundlagen Betriebssysteme und Systemsoftware (GBS), allerdings könnten
              vereinzelnt Theoriefragen zu grundlegenden Inhalten gestellt werden}
              \item Interrupts, Traps, Exceptions
              \item virtuelle Speicherverwaltung
          \end{enumerate}
          Aufgaben: 4.1, 4.3, 4.4, eigene Problemstellungen ausdenken und in Assembly implementieren!
    \item Boolesche Algebra und kombinatorische Schaltungen
          \begin{enumerate}[i.]
              \item Gesetze der booleschen Algebra
              \item Gattersymbole
              \item Addiererschaltungen
          \end{enumerate}
          Aufgaben: 5.1*, 5.2*
    \item Arithmetische und sequentielle Schaltungen
          \begin{enumerate}[i.]
              \item Multiplizierer
              \item Multiplexer
              \item ALU 
              \item Flipflops und Latches
              \item Rückkopplungen
          \end{enumerate}
          Aufgaben: 6.1*, 6.3
    \item Maschinensprache und Prozessorschaltbild
          \begin{enumerate}[i.]
              \item Instruktionskodierung
              \item RISC-V-Befehlstypen
              \item Single-Cycle-Prozessor, insbesondere Signalbelegungen
          \end{enumerate}
          Aufgaben: 7.2*, 7.3*, 7.4*
    \item Automaten und Multi-Cycle-Prozessor
          \begin{enumerate}[i.]
              \item Automaten, Statuskodierungen (One-Hot, Binär)
              \item Multi-Cycle-Prozessor, insbesondere Signalbelegungen
          \end{enumerate}
          Aufgaben: 8.1, 8.2, 8.3
    \item Pipelining
          \begin{enumerate}[i.]
              \item Pipelined Prozessor
              \item Datenabhängigkeiten, Steuerungs- und Datenkonflikte
              \item Hazard Unit: Forwarding, Stalling
              \item Branch prediction, Superskalarität, Out-of-Order-Execution, Register Renaming
          \end{enumerate}
          Aufgaben: 9.2, 9.3*
    \item Caches
          \begin{enumerate}[i.]
              \item Speicherhierarchie
              \item DRAM/SRAM
              \item zeitliche und räumliche Lokalität
              \item Ersetzungsstrategien
              \item Terminologie: Hit/Miss/Hit Latency/\ldots
              \item Voll-assoziative, Direct-mapped und Mengenassoziative Caches
          \end{enumerate}
          Aufgaben: 10.1*, 10.2*
    \item Logiksynthese und Optimierung
          \begin{enumerate}[i.]
              \item K-Maps 
              \item Binary Decision Diagrams (BDDs)
              \item ITE
              \item Konjunktive und Disjunktive Normalform
          \end{enumerate}
          Aufgaben: 11.1*, 11.2, 11.3*
    \item AIGs und SAT-Solving
          \begin{enumerate}[i.]
              \item And-Inverter-Graphen (Konstruktion und Optimierung)
              \item DPLL-Algorithmus
              \item Konfliktgraphen und gelernte Klauseln
              \item Tseitin-Transformation
          \end{enumerate}
          Aufgaben: 12.1, 12.2
    \item Parallelisierung
          \begin{enumerate}[i.]
              \item Flynn's Classification
              \item SIMD
              \item Multithreading
              \item Mehrkernsysteme
              \item MSI/MESI 
              \item Synchronisierung
              \item Kategorisierung von Parallelismen
              \item Speedup nach Amdahl und Gustafson
              \item Roofline-Modell
          \end{enumerate}
          Aufgaben: 13.1*, 13.2*, 13.3
    \item Ein- und Ausgabe
    \begin{enumerate}[i.]
        \item Netzwerktopologien
        \item DMA
    \end{enumerate}
\end{enumerate}
\vspace{3cm}
\begin{center}
\Large Viel Erfolg bei der Klausur!
\end{center}
\end{document}