%%
% This is an Overleaf template for presentations
% using the TUM Corporate Desing https://www.tum.de/cd
%
% For further details on how to use the template, take a look at our
% GitLab repository and browse through our test documents
% https://gitlab.lrz.de/latex4ei/tum-templates.
%
% The tumbeamer class is based on the beamer class.
% If you need further customization please consult the beamer class guide
% https://ctan.org/pkg/beamer.
% Additional class options are passed down to the base class.
%
% If you encounter any bugs or undesired behaviour, please raise an issue
% in our GitLab repository
% https://gitlab.lrz.de/latex4ei/tum-templates/issues
% and provide a description and minimal working example of your problem.
%%

\PassOptionsToClass{onlytextwidth}{beamer}

\documentclass[
  german,            % define the document language (english, german)
  aspectratio=169,    % define the aspect ratio (169, 43)
  % handout=2on1,       % create handout with multiple slides (2on1, 4on1)
  % partpage=false,     % insert page at beginning of parts (true, false)
  % sectionpage=true,   % insert page at beginning of sections (true, false)
]{tumbeamer}


% load additional packages
\usepackage{tikz}
\usepackage{circuitikz}
\usepackage{url}
\usepackage{hyperref}
\usepackage{pgf}
\usepackage{pgfplots}
\usepackage{babel}[ngerman]
\usepackage{csquotes}[autostyle]
\usepackage[useregional]{datetime2}
\usepackage{float}
\usepackage{graphicx}
\usepackage{amsmath}
\usepackage{xcolor}
\usepackage[cache=true]{minted}
\usemintedstyle{borland}
\usepackage{listings}



% tikz  
\usetikzlibrary{fit, matrix, calc, arrows, arrows.meta, positioning, patterns, patterns.meta, bending, overlay-beamer-styles, shapes, shapes.geometric, shapes.misc, backgrounds}

% https://tex.stackexchange.com/a/7045
\newcommand*\circled[1]{\tikz[baseline=(char.base)]{
		\node[shape=circle,draw,inner sep=2pt, font=\scriptsize] (char) {#1};}}

\newcommand*\colorcirc[1]{\tikz[baseline=(char.base)]{
		\node[shape=circle,fill=#1,inner sep=2pt] {};}}
	
\newcommand\n[1]{\mkern1mu\overline{\mkern-1mu#1}}

% requires circuitikz >= 1.1.0
% for distros with older distributions, install TeX Live manually
% instead of using your package manager
% see: https://tug.org/texlive/quickinstall.html
\ctikzset{logic ports=ieee}

% minted
\setminted{
	fontsize=\small, 
	frame=none,
	breaklines=true,
}

\captionsetup{labelformat=empty}

% image path
\graphicspath{ {./resources/} }

% beamer
\setbeamercolor{footnote}{fg=black}
\setbeamercolor{footnote mark}{fg=black}
\renewcommand{\thempfootnote}{\arabic{mpfootnote}}

% presentation metadata
\title{Übung 06: Kombinatorische \\Schaltungen}
\subtitle{Einführung in die Rechnerarchitektur}
\author{Niklas Ladurner}

\institute{\theChairName\\\theDepartmentName\\\theUniversityName}
\date{\DTMdisplaydate{2025}{11}{21}{-1}}

\footline{\insertauthor~|~\insertshorttitle~|~\insertshortdate}


% macro to configure the style of the presentation
\TUMbeamersetup{
  title page = TUM tower,         % style of the title page
  part page = TUM toc,            % style of part pages
  section page = TUM toc,         % style of section pages
  content page = TUM more space,  % style of normal content pages
  tower scale = 1.0,              % scaling factor of TUM tower (if used)
  headline = TUM threeliner,      % which variation of headline to use
  footline = TUM default,         % which variation of footline to use
  % configure on which pages headlines and footlines should be printed
  headline on = {title page},
  footline on = {every page, title page=false},
}

\begin{document}

\maketitle

\begin{frame}[c, fragile]{}{}
  \begin{center}
    \LARGE  Keine Garantie für die Richtigkeit der Tutorfolien.

    \Large Bei Unklarheiten/Unstimmigkeiten haben VL/ZÜ-Folien recht!
  \end{center}
\end{frame}

\begin{frame}[c, fragile]{Boolesche Funktionen}{}
  \centering
  \begin{columns}[T]
    \begin{column}{0.3\textwidth}
      \centering
      OR-Gatter

      \vspace{0.2cm}

      \begin{circuitikz}
        \draw (0,0) node[or port] (or) {};
        \draw (or.in 1) node[left] {$A$};
        \draw (or.in 2) node[left] {$B$};
        \draw (or.out) -- ++(0.5,0) node[right] {$A \lor B$};
      \end{circuitikz}

      \vspace{-0.3cm}

      \[
        \begin{array}{|c|c|c|}
          \hline
          A & B & A \lor B \\
          \hline
          0 & 0 & 0        \\
          0 & 1 & 1        \\
          1 & 0 & 1        \\
          1 & 1 & 1        \\
          \hline
        \end{array}
      \]

    \end{column}

    \begin{column}{0.3\textwidth}
      \centering
      AND-Gatter

      \vspace{0.2cm}

      \begin{circuitikz}
        \draw (0,0) node[and port] (and) {};
        \draw (and.in 1) node[left] {$A$};
        \draw (and.in 2) node[left] {$B$};
        \draw (and.out) -- ++(0.5,0) node[right] {$A \land B$};
      \end{circuitikz}

      \vspace{-0.3cm}

      \[
        \begin{array}{|c|c|c|}
          \hline
          A & B & A \land B \\
          \hline
          0 & 0 & 0         \\
          0 & 1 & 0         \\
          1 & 0 & 0         \\
          1 & 1 & 1         \\
          \hline
        \end{array}
      \]

    \end{column}

    \begin{column}{0.3\textwidth}
      \centering
      NOT-Gatter

      \vspace{0.2cm}

      \begin{circuitikz}
        \draw (0,0) node[not port] (not) {};
        \draw (not.in) -- ++(-0.5,0) node[left] {$A$};
        \draw (not.out) -- ++(0.5,0) node[right] {$\lnot A$};
      \end{circuitikz}

      \vspace{-0.3cm}

      \[
        \begin{array}{|c|c|}
          \hline
          A & \lnot A \\
          \hline
          0 & 1       \\
          1 & 0       \\
          \hline
        \end{array}
      \]

    \end{column}

  \end{columns}
\end{frame}

\begin{frame}[c, fragile]{Boolesche Funktionen}{}
  \centering
  \begin{columns}[T]
    \begin{column}{0.5\textwidth}
      \centering
      XOR-Gatter

      \vspace{0.2cm}

      \begin{circuitikz}
        \draw (0,0) node[xor port] (or) {};
        \draw (or.in 1) node[left] {$A$};
        \draw (or.in 2) node[left] {$B$};
        \draw (or.out) -- ++(0.5,0) node[right] {$A \oplus B$};
      \end{circuitikz}

      \vspace{-0.3cm}

      \[
        \begin{array}{|c|c|c|}
          \hline
          A & B & A \oplus B \\
          \hline
          0 & 0 & 0        \\
          0 & 1 & 1        \\
          1 & 0 & 1        \\
          1 & 1 & 0        \\
          \hline
        \end{array}
      \]

    \end{column}

    \begin{column}{0.5\textwidth}
      \centering
      XNOR-Gatter

      \vspace{0.2cm}

      \begin{circuitikz}
        \draw (0,0) node[xnor port] (or) {};
        \draw (or.in 1) node[left] {$A$};
        \draw (or.in 2) node[left] {$B$};
        \draw (or.out) -- ++(0.5,0) node[right] {$A\leftrightarrow B$};
      \end{circuitikz}

      \vspace{-0.3cm}

      \[
        \begin{array}{|c|c|c|}
          \hline
          A & B & A\leftrightarrow B\\
          \hline
          0 & 0 & 1        \\
          0 & 1 & 0        \\
          1 & 0 & 0        \\
          1 & 1 & 1        \\
          \hline
        \end{array}
      \]
    \end{column}


  \end{columns}
\end{frame}

\begin{frame}[c, fragile]{Definitionen (1)}{}
	\begin{block}{Funktionale Vollständigkeit}
		Eine Menge $M$ boolescher Operatoren heißt funktional vollständig, falls alle booleschen Funktionen $f_i\in\mathcal{F}$ als Kombination von Operatoren aus $M$ darstellbar sind. Die Menge $\{\wedge, \neg\}$ ist funktional vollständig.
	\end{block}
	\vfill{}
	\begin{itemize}
		\item Struktur zum Beweis, dass eine Menge $M$ funktional vollständig ist:
		\begin{itemize}
			\item Zeige, dass $\neg x$ durch Operatoren aus $M$ dargestellt werden kann
			\item Zeige, dass $x\wedge y$ durch Operatoren aus $M$ dargestellt werden kann
		\end{itemize}
		\item Gegenbeweis deutlich schwieriger: Finden einer booleschen Funktion  mit $n\ge1$ Variablen, die nicht dargestellt werden kann (bspw. $\n{x}$)
	\end{itemize}
\end{frame}

\begin{frame}[c, fragile]{Definitionen  (2)}{}
	\begin{block}{Dualität}
		Zu einer gegebenen boolesche Formel $f$ in Negationsnormalform erhält man den dazugehörigen dualen Ausdruck $f^D$ durch Ersetzung: $\{0\mapsto 1; 1\mapsto 0; \wedge\mapsto\vee; \vee\mapsto\wedge\}$. Negationen bleiben davon unberührt!
	\end{block}
	\vfill{}
	\begin{block}{Beispiele}
		\begin{enumerate}
			\item $f=(x \vee y) \wedge 1\leadsto f^D=(x\wedge y)\vee 0$
			\item $g=(a \wedge b\wedge c) \vee (\n{a} \wedge (0\vee \n{b}))\leadsto g^D=(a \vee b\vee c) \wedge (\n{a} \vee (1\wedge \n{b}))$
		\end{enumerate}
	\end{block}
\end{frame}

\begin{frame}[c, fragile]{Definitionen (3)}{}
	\begin{block}{Negationstheorem}
		Es gilt, dass $\n{f(x_1, x_2, \ldots, x_n, 0, 1, \wedge, \vee)} \equiv f(\n{x_1}, \n{x_2}, \ldots, \n{x_n}, 1, 0, \vee, \wedge)$.
		
		Achtung: Anders als bei der Dualität werden hier auch die Variablen negiert!
	\end{block}
	\vfill{}
	\begin{block}{Beispiele}
		\begin{enumerate}
			\item $\n{x\vee y} \equiv (\n{x} \wedge \n{y})$ (klassischer De Morgan)
			\item $\n{a \vee (b\wedge 1) \vee (c \vee 0)} \equiv \n{a} \wedge (\n{b}\vee 0) \wedge (\n{c} \wedge 1)$
		\end{enumerate}
	\end{block}
\end{frame}

\begin{frame}[c, fragile]{Gesetze der booleschen Algebra\footnote{In ERA werden sowohl die Schreibweisen $\wedge/\vee$ als auch $\cdot / +$ akzeptiert, solange sie einheitlich verwendet werden.}}{}
	\begin{center}
		\begin{tabular}{lcc}
			Bezeichnung & Gesetz & duales Gesetz  \\\hline
			Involution & $\n{\n{x}}\equiv x$ & --\\
			Idempotenz & $x+x\equiv x$ & $x\cdot x\equiv x$ \\
			Neutralität & $x+0\equiv x$ & $x\cdot 1\equiv x$  \\
			Extremalgesetz & $x+1\equiv 1$ & $x\cdot 0\equiv 0$  \\
			Komplementärgesetz & $x+\n{x}\equiv 1$ & $x\cdot\n{x}\equiv 0$  \\\pause
			Assoziativität &$x + (y+z)\equiv (x + y)+z$ & $x\cdot(y\cdot z)\equiv (x\cdot y)\cdot z$ \\
			Kommutativität &$x + y\equiv y+x$ & $x\cdot y\equiv y\cdot x$ \\
			Distributivität & $x+(y\cdot z)\equiv (x+y)\cdot (x+z)$  & $x\cdot(y+z)\equiv (x\cdot y)+ (x\cdot z)$ \\
			De Morgan &$\n{x+ y} \equiv (\n{x} \cdot \n{y})$ & $\n{x\cdot y} \equiv (\n{x} + \n{y})$ \\
			Absorption & $x + (x\cdot y) \equiv x$ &$x \cdot (x+y) \equiv x$  \\
		\end{tabular}
	\end{center}
\end{frame}

\begin{frame}[c, fragile]{Normalformen}{}
	\begin{columns}[c]
		\begin{column}{0.5\textwidth}
			\vspace*{-1cm}
			\begin{itemize}
				\item Konjunktive Normalform (OR in den Klammern, AND dazwischen): $(x+y) \cdot (x+\n{y})$
				\item Disjunktive Normalform (AND in den Klammern, OR dazwischen): $(x\cdot y) + (x\cdot\n{y})$
				\item KNF bei wenigen Nullen im Ergebnis, DNF bei wenigen Einsen im Ergebnis 
			\end{itemize}
		\end{column}%
		\begin{column}{0.5\textwidth}
			\vspace*{0.5cm}
			\includegraphics[width=\textwidth]{w06_knf_dnf.pdf}
			\centering\scriptsize{Quelle: \href{https://de.wikipedia.org/wiki/Konjunktive_Normalform}{Wikipedia}}
		\end{column}
	\end{columns}
\end{frame}

\begin{frame}[c, fragile]{}{}
  \begin{center}
    \LARGE Fragen?
  \end{center}
\end{frame}

\begin{frame}[c, fragile]{Links}{}
  \begin{itemize}
    \item Zulip: \href{https://zulip.in.tum.de/#narrow/channel/3255-ERA-Tutorium-.E2.80.93-Mi-1600-3}{\enquote{ERA Tutorium -- Mi-1600-3}}
    bzw. \href{https://zulip.in.tum.de/#narrow/channel/3264-ERA-Tutorium-.E2.80.93-Fr-1500-1}{\enquote{ERA Tutorium -- Fr-1500-1}}
    \item \href{https://www.moodle.tum.de/course/view.php?id=111440}{ERA-Moodle-Kurs}
    \item \href{https://artemis.tum.de/courses/516}{ERA-Artemis-Kurs}
    \item \href{https://www.elektronik-kompendium.de/sites/dig/2609191.htm}{Elektronik-Kompendium zu logischen Grundschaltungen}
    \item \href{https://github.com/hneemann/Digital}{GitHub: Digital}
  \end{itemize}
\end{frame}

\maketitle

\end{document}
