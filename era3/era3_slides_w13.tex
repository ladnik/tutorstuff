%%
% This is an Overleaf template for presentations
% using the TUM Corporate Desing https://www.tum.de/cd
%
% For further details on how to use the template, take a look at our
% GitLab repository and browse through our test documents
% https://gitlab.lrz.de/latex4ei/tum-templates.
%
% The tumbeamer class is based on the beamer class.
% If you need further customization please consult the beamer class guide
% https://ctan.org/pkg/beamer.
% Additional class options are passed down to the base class.
%
% If you encounter any bugs or undesired behaviour, please raise an issue
% in our GitLab repository
% https://gitlab.lrz.de/latex4ei/tum-templates/issues
% and provide a description and minimal working example of your problem.
%%

\PassOptionsToClass{onlytextwidth}{beamer}

\documentclass[
german,            % define the document language (english, german)
aspectratio=169,    % define the aspect ratio (169, 43)
% handout=2on1,       % create handout with multiple slides (2on1, 4on1)
% partpage=false,     % insert page at beginning of parts (true, false)
% sectionpage=true,   % insert page at beginning of sections (true, false)
]{tumbeamer}


% load additional packages
\usepackage{tikz}
\usepackage{circuitikz}
\usepackage{url}
\usepackage{hyperref}
\usepackage{pgf}
\usepackage{pgfplots}
\usepackage{babel}[ngerman]
\usepackage{csquotes}[autostyle]
\usepackage[useregional]{datetime2}
\usepackage{float}
\usepackage{graphicx}
\usepackage{amsmath}
\usepackage{amssymb}
\usepackage{xcolor}
\usepackage[cache=true]{minted}
\usemintedstyle{borland}
\usepackage{listings}
\usepackage{tikz-timing}
\usepackage{ifthen}
\usepackage{karnaugh-map}

% tikz
\usetikzlibrary{overlay-beamer-styles}
\usetikzlibrary{arrows,backgrounds,positioning,shapes,patterns,patterns.meta,matrix,arrows,shapes.geometric,matrix,fit,calc,automata}

% requires circuitikz >= 1.1.0
% for distros with older distributions, install TeX Live manually
% instead of using your package manager
% see: https://tug.org/texlive/quickinstall.html
\ctikzset{logic ports=ieee}

% minted
\setminted{
	fontsize=\small, 
	frame=none,
	breaklines=true,
}

% image path
\graphicspath{ {./resources/} }

% beamer
\setbeamercolor{footnote}{fg=black}
\setbeamercolor{footnote mark}{fg=black}

% additional commands
\newcommand{\n}[1]{\overline{#1}}

% presentation metadata
\title{Übung 13: Optimierung}
\subtitle{Einführung in die Rechnerarchitektur}
\author{Niklas Ladurner}

\institute{\theChairName\\\theDepartmentName\\\theUniversityName}
\date{\DTMdisplaydate{2026}{01}{23}{-1}}

\footline{\insertauthor~|~\insertshorttitle~|~\insertshortdate}


% macro to configure the style of the presentation
\TUMbeamersetup{
  title page = TUM tower,         % style of the title page
  part page = TUM toc,            % style of part pages
  section page = TUM toc,         % style of section pages
  content page = TUM more space,  % style of normal content pages
  tower scale = 1.0,              % scaling factor of TUM tower (if used)
  headline = TUM threeliner,      % which variation of headline to use
  footline = TUM default,         % which variation of footline to use
  % configure on which pages headlines and footlines should be printed
  headline on = {title page},
  footline on = {every page, title page=false},
}

\setbeamercolor{enumerate item}{fg=black}

\begin{document}

\maketitle

\begin{frame}[c, fragile]{}{}
	\begin{center}
		\LARGE  Keine Garantie für die Richtigkeit der Tutorfolien.

		\Large Bei Unklarheiten/Unstimmigkeiten haben VL/ZÜ-Folien recht!
	\end{center}
\end{frame}

\begin{frame}[fragile, c]{Logiksynthese}{}
	\begin{itemize}
		\item Logiksynthese: Realisierung boolsche Funktion in Hardware
		\item naive Synthese (direkte Übertragung der Wahrheitstabelle) nicht skalierbar
		\item verschiedene Verfahren zur Optimierung und Reduktion von Funktionen auf ihr Minimalpolynom $\rightarrow$ optimale Schaltung
	\end{itemize}
	\begin{block}{Minimalpolynom}
		Ein Polynom $p$ ist Minimalpolynom einer booleschen Funktion $f$, falls  $\psi(p) \equiv f$ (d.h. $p$ eine Formel für $f$ ist) und es keine weitere Vereinfachungen gibt.
	\end{block}
\end{frame}

\begin{frame}[fragile, c]{Karnaugh-Veitch-Diagramme\footnote{Oft auch als K-Maps bezeichnet}}{}
\begin{columns}[c]
\begin{column}{0.6\textwidth}
	\begin{itemize}
		\item rechteckiges Schema, in dem alle Literalkombinationen (positiv und negativ) vorkommen
		\item nebeneinander liegende Zeilen/Spalten dürfen sich immer nur in 1 Bit unterscheiden (Gray-Code)!
		\item Zusammenfassen von Einsen in $2^n$-Blöcken, Don't Care können als $0$ oder $1$ gewählt werden.
		\item jedes maximal große Päckchen steht für einen Primimplikanten der Funktion $\rightarrow$ minimale Mengen von Päckchen die alle $1$en abdeckt ergibt ein Minimalpolynom
	\end{itemize}
\end{column}
\begin{column}{0.4\textwidth}
\begin{center}
\resizebox{!}{0.6\textheight}{
	\begin{karnaugh-map}(label=corner,implicantcolors={TUMExtRed,TUMExtLime,TUMOrange,TUMBlue,TUMExtViolet})[4][4][1][d][c][b][a]
	\maxterms{3,6,7,12,9} % zeroes
	\minterms{4,5,15,14,8,11,10,13,2} % ones
	\implicant{0}{5}
	\implicant{15}{10}
	\implicant{5}{13}
	\implicantcorner
	\autoterms[-]
	\end{karnaugh-map}
}\\
$f=\textcolor{TUMExtRed}{\n{a}\n{c}}+\textcolor{TUMBlue}{\n{b}\n{d}}+\textcolor{TUMExtLime}{ac} +\textcolor{TUMOrange}{b\n{c}d}$
\end{center}
\end{column}
\end{columns}

\end{frame}

\begin{frame}[fragile]{Logik-Hazards}{}
\begin{columns}[c]
\begin{column}{0.5\textwidth}
	\begin{itemize}
		\item Glitches (kurzzeitige Signaländerung) am Ausgang, die durch Gatterlaufzeit auftritt
		\item kann zu Problemen führen (z.B. Oszillation)
		\item in K-Map: Zwei nebeneinanderliegende Einsen, die nicht durch ein gemeinsames Päckchen abgedeckt sind
		\item Lösung: Hinzufügen von zusätzlichem Päckchen $\rightarrow$ kein Minimalpolynom!
	\end{itemize}
\end{column}
\begin{column}{0.5\textwidth}
\vspace{-0.5cm}
\begin{center}
\resizebox{0.75\textwidth}{!}{
	\begin{karnaugh-map}(label=corner,implicantcolors={TUMExtRed,TUMExtLime,TUMOrange,TUMBlue,TUMExtViolet})[4][2][1][c][b][a]
	%\maxterms{} % zeroes
	%\indeterminants{}
	\minterms{0,1,5,7} % ones
	\autoterms[0]
	\implicant{0}{1}
	\implicant{5}{7}
	\only<3->{\implicant{1}{5}}
	\end{karnaugh-map}
}
\resizebox{0.75\textwidth}{!}{
	\begin{circuitikz}
		\node[or port, number inputs=3, no input leads] (or) at (0,0) {};

		\node[and port, number inputs=2, left=of or.in 1, anchor=out, yshift=1cm] (and1) {};
		\node at (and1.bin 1) [notcirc, left]{};
		\node at (and1.bin 2) [notcirc, left]{};

		\node[and port, number inputs=2, left=of or.in 2, anchor=out, draw=white] (and2phantom) {};

		\node[and port, number inputs=2, left=of or.in 3, anchor=out, yshift=-1cm] (and3) {};

		\draw (and1.in 1) -- ++(-.75, 0) node[left] (a) {$a$};
		\draw (and1.in 2) -- ++(-.75, 0) node[left] (b) {$b$};

		\draw (and2phantom.in 2)++(-0.25, 0) -- ++(-.5, 0) node[left] (c) {$c$};

		\draw (a) -- (and1.in 1);
		\draw (b) -- (and1.in 2);


		\draw (c) -- ++(0.75, 0) |- (and3.in 1);
		\draw (a) -- ++(0.5, 0) node[circle,fill=black,inner sep=0pt,minimum size=3pt] {}  |- (and3.in 2);

		\draw (and1.out) |- (or.bin 1);
		\draw (and3.out) |- (or.bin 3);

		\draw (or.out) -- ++(.5, 0) node[right] (y) {$y$};

		\only<2>{
			\draw[very thick, red] (a) -- (and1.bin 1) -- (and1.bout) -- (and1.out) |- (or.bin 1) -- (or.bout) -- (y);
		}

		\only<3->{
			\node[and port, number inputs=2, left=of or.in 2, anchor=out, TUMOrange] (and2) {};
			\node at (and2.bin 1) [notcirc, left, TUMOrange]{};
			\draw (b) -- ++(0.75, 0) node[circle,fill=black,inner sep=0pt,minimum size=3pt] {}  |- (and2.in 1);
			\draw (c) ++(0.75, 0) node[circle,fill=black,inner sep=0pt,minimum size=3pt] {};
			\draw (c) -- (and2.in 2);
			\draw (and2.out) |- (or.bin 2);
		}
	\end{circuitikz}
}
\end{center}
\end{column}
\end{columns}

\end{frame}

% stolen from sheet u12
\tikzset{
	zeroarrow/.style = {-stealth,dashed},
	onearrow/.style = {-stealth,solid},
	c/.style = {circle,draw,solid,minimum width=2em,
			minimum height=2em},
	r/.style = {rectangle,draw,solid,minimum width=2em,
			minimum height=2em},
	e/.style = {draw=none,circle, minimum width=2em,minimum height=2em}
}
\begin{frame}[c, fragile]{Binary Decision Diagrams}{}
	\begin{columns}[c]
		\begin{column}{0.5\textwidth}
			\begin{itemize}
				\item Darstellung einer boolschen Funktion als gerichteter azyklischer Graph (DAG)
				\item Knoten repräsentieren Teilfunktionen, 2 ausgehende Kanten: $0$ (low), $1$ (high)
				\item Aufbau bspw. mittels Shannon-Zerlegung: $f(x_0, x_1) \rightarrow f_{x_0=0}(x_1), f_{x_0=1}(x_1)$
				\item \textbf{ROBDDs sind kanonisch (eindeutig)}
			\end{itemize}
		\end{column}
		\begin{column}{0.5\textwidth}
			\begin{center}
				\resizebox{!}{0.8\textheight}{
					\begin{tikzpicture}[node distance=1cm and 1cm]\footnotesize
						\node[c] (x1) {$x_1$};
						\node[] (t) [above=of x1, yshift=-0.8cm] {$\n{x}_1x_2\n{x}_3+x_1\n{x}_2\n{x}_3+x_1x_2x_3$};
						\node[c] (x2l) [below left=of x1] {$x_2$};
						\node[c] (x2r) [below right=of x1] {$x_2$};

						\node[c] (x3l) [below=of x2l] {$x_3$};
						\node[c] (x3r) [below=of x2r] {$x_3$};

						\node[r] (zero) [below=of x3r, xshift=-0.5cm] {0};
						\node[r] (one) [below=of x3l, xshift=0.5cm] {1};

						\draw[zeroarrow] (x1) -- (x2l) node [midway, fill=none, above left] {$x_2\n{x}_3$};
						\draw[onearrow] (x1) -- (x2r)  node [near end, fill=none, above right, align=center] {$\n{x}_2\n{x}_3+$\\$x_2x_3$};

						\draw[zeroarrow] (x2l) to [out=0,in=180] (zero);
						\draw[onearrow] (x2l) -- (x3l) node [midway, fill=none, left] {$\n{x}_3$};

						\draw[zeroarrow] (x2r) -- (x3l)  node [near end, fill=none, above left] {$\n{x}_3$};
						\draw[onearrow] (x2r)  -- (x3r)  node [midway, fill=none,right] {$x_3$};

						\draw[zeroarrow] (x3l) -- (one);
						\draw[onearrow] (x3l) -- (zero);
						\draw[zeroarrow] (x3r) -- (zero);
						\draw[onearrow] (x3r) -- (one);
					\end{tikzpicture}
				}

				\scriptsize Variablenordnung: $x_1\prec x_2\prec x_3$
			\end{center}
		\end{column}
	\end{columns}
\end{frame}

\begin{frame}[c, fragile]{BDDs: Reduktionen}
	\begin{columns}[c]
		\begin{column}{0.5\textwidth}
			\begin{center}
				\textbf{I-Reduktion}\\
				Zusammenführung isomorpher Knoten
				\vspace{\baselineskip}
				\resizebox{!}{0.5\textheight}{
					\begin{tikzpicture}
						\node[e] (rl) {};
						\node[e] (rm) [right=1em of rl]{};
						\node[e] (rr) [right=1em of rm]{};
						\node[c] (al) [below=of rl] {$a$};
						\node[c] (ar) [below=of rr] {$a$};

						\node[c] (bl) [below=of al] {$b$};
						\node[c] (br) [below=of ar] {$b$};

						\draw[zeroarrow] (rl) -- (al);
						\draw[zeroarrow] (rm) -- (ar);
						\draw[onearrow] (rr) -- (ar);

						\draw[zeroarrow] (al) -- (bl);
						\draw[zeroarrow] (ar) -- (bl);

						\draw[onearrow] (al) -- (br);
						\draw[onearrow] (ar) -- (br);

						\node[e] (rl2) [right=4em of rr]{};
						\node[e] (rm2) [right=1em of rl2]{};
						\node[e] (rr2) [right=1em of rm2]{};
						\node[c] (al2) [below=of rl2] {$a$};

						\node[c] (bl2) [below=of al2] {$b$};
						\node[c] (br2) [below=of al2, xshift=2cm] {$b$};

						\draw[zeroarrow] (rl2) -- (al2);
						\draw[zeroarrow] (rm2) -- (al2);
						\draw[onearrow] (rr2) -- (al2);

						\draw[zeroarrow] (al2) -- (bl2);
						\draw[zeroarrow] (al2) -- (bl2);

						\draw[onearrow] (al2) -- (br2);
						\draw[onearrow] (al2) -- (br2);

						\draw[-Latex, thick] (ar)++(2em,0) -- ++(2em,0);
					\end{tikzpicture}
				}
			\end{center}
		\end{column}
		\begin{column}{0.5\textwidth}
			\begin{center}
				\textbf{S-Reduktion}\\
				Entfernen von überflüssigen Knoten
				\vspace{\baselineskip}
				\resizebox{!}{0.5\textheight}{
					\begin{tikzpicture}
						\node[e] (rl) {};
						\node[e] (rm) [right=1em of rl]{};
						\node[e] (rr) [right=1em of rm]{};
						\node[c] (al) [below=of rm] {$a$};
						\node[c] (bl) [below=of al] {$b$};

						\draw[zeroarrow] (rl) -- (al);
						\draw[zeroarrow] (rm) -- (al);
						\draw[onearrow] (rr) -- (al);

						\draw[zeroarrow] (al) to[bend right] (bl);
						\draw[onearrow] (al) to[bend left] (bl);

						\node[e] (rl2) [right=0em of rr]{};
						\node[e] (rm2) [right=1em of rl2]{};
						\node[e] (rr2) [right=1em of rm2]{};

						\node[e] (al2) [below=of rm2] {};
						\node[c] (bl2) [below=of al2] {$b$};

						\draw[zeroarrow] (rl2) -- (bl2);
						\draw[zeroarrow] (rm2) -- (bl2);
						\draw[onearrow] (rr2) -- (bl2);

						\draw[-Latex, thick] (al)++(3em,0) -- ++(2em,0);
					\end{tikzpicture}
				}
			\end{center}
		\end{column}
	\end{columns}
\end{frame}

\begin{frame}[c, fragile]{ITE}
	\begin{columns}[c]
		\begin{column}{0.5\textwidth}
			\begin{itemize}
				\item Ternärer Operator $\textrm{ITE}(a, b, c)\equiv ab + \n{a}c$; falls $a$, dann $b$, sonst $c$
				\item alle booleschen Operatoren können als ITE dargestellt werden, z.B.
				      \begin{enumerate}
					      \item $\textrm{NOT}(f) \equiv \textrm{ITE}(f, 0, 1)$
					      \item $\textrm{OR}(f,g) \equiv \textrm{ITE}(f, 1, g)$
					      \item $\textrm{AND}(f,g) \equiv \textrm{ITE}(f, g, 0)$
				      \end{enumerate}
				\item Zusammenhang mit BDDs: High-Kind im True-Fall, Low-Kind im False-Fall
			\end{itemize}
		\end{column}
		\begin{column}{0.5\textwidth}
			\begin{center}
				\begin{tikzpicture}
					\def\smallspace{0.75cm}
					\def\bigspace{6cm}
					\def\margin{0.75cm}
					\def\labelspacing{0cm}

					\node[c] (x1) {$a$};
					\node[above=\labelspacing of x1] {$f$};

					\node[c] (x2l) [below=\smallspace of x1, xshift=-2*\smallspace] {$b$};
					\node[above=\labelspacing of x2l] {$f_{\n{a}}$};
					\node[c] (x2r) [below=\smallspace of x1, xshift=2*\smallspace] {$b$};
					\node[above=\labelspacing of x2r] {$f_{a}$};

					\node[c] (x3rl) [below=\smallspace of x2r, xshift=-2*\smallspace] {$c$};
					\node[above=\labelspacing of x3rl] {$f_{a\n{b}}$};
					\node[c] (x3rr) [below=\smallspace of x2r, xshift=2*\smallspace] {$c$};
					\node[above=\labelspacing of x3rr] {$f_{ab}$};

					\draw[zeroarrow](x1) -- (x2l);
					\draw[onearrow](x1) -- (x2r);

					\draw[zeroarrow](x2r) -- (x3rl);
					\draw[onearrow](x2r) -- (x3rr);

					\draw[zeroarrow] (x2l) -- ($(x2l) + (-\margin, -\margin)$) node[below] {$\ldots$};
					\draw[onearrow] (x2l) -- ($(x2l) + (\margin, -\margin)$) node[below] {$\ldots$};

					\draw[zeroarrow] (x3rl) -- ($(x3rl) + (-\margin, -\margin)$) node[below] {$\ldots$};
					\draw[onearrow] (x3rl) -- ($(x3rl) + (\margin, -\margin)$) node[below] {$\ldots$};
					\draw[zeroarrow] (x3rr) -- ($(x3rr) + (-\margin, -\margin)$) node[below] {$\ldots$};
					\draw[onearrow] (x3rr) -- ($(x3rr) + (\margin, -\margin)$) node[below] {$\ldots$};
				\end{tikzpicture}

				$f\equiv\textrm{ITE}(a, f_{a}, f_{\n{a}})$ $\equiv\textrm{ITE}(a, \textrm{ITE}(b, f_{ab}, f_{a\n{b}}), f_{\n{a}})$
			\end{center}
		\end{column}
	\end{columns}

\end{frame}

\begin{frame}[c, fragile]{BDDs: ITE-Algorithmus}
	\tikzset{
		highlight/.style={draw=TUMBlue, thick, fill=TUMBlue, fill opacity=0.35, text opacity=1},
		ahighlight/.style={draw=TUMBlue, thick},
		highlight on/.style={alt=#1{highlight}{}},
		ahighlight on/.style={alt=#1{ahighlight}{}},
	}

	\only<1>{
		Am Beispiel der Verknüpfung von zwei BDDs $A, B$ mittels $\oplus$:
		\begin{enumerate}
			\item Bezeichne alle Knoten mit einem eindeutigen Bezeichner
			\item Erzeuge im Ergebnis-BDD einen Knoten $A_0 \oplus B_0$
			\item Solange noch nicht alle Knoten besucht wurden:
			      \begin{enumerate}
				      \item Falls Terminalfall (d.h. $A_i$ und $B_i$ sind Terminalknoten), Rückgabewert berechnen und im Ergebnis-BDD an der aktuellen Stelle einen neuen Terminalknoten $A_i \oplus B_i$  erzeugen
				      \item Top-Variable $t$ anhand der Variablenordnung bestimmen
				      \item Für alle mit $t$ bezeichneten Knoten: Abstieg ins Low-Kind, erzeuge im Ergebnis-BDD an der aktuellen Stelle den Kindknoten $A_{i, t=0} \oplus B_{i, t=0}$
				      \item Für alle mit $t$ bezeichneten Knoten: Abstieg ins High-Kind, erzeuge im Ergebnis-BDD an der aktuellen Stelle den Kindknoten $A_{i, t=1} \oplus B_{i, t=1}$
			      \end{enumerate}
			\item Eventuell I- und S-Reduktionen anwenden
		\end{enumerate}
	}
	\only<2-12>{
		\begin{center}
			\begin{tikzpicture}

				\def\smallspace{0.75cm}
				\def\bigspace{3cm}
				\def\margin{0.75cm}
				\def\labelspacing{0cm}

				% f
				\node[c, highlight on=<3>, highlight on=<9>] (x1) {$a$};
				\node[right=\labelspacing of x1] {$A_1$};

				\node[c] (x2l) [below=\smallspace of x1, xshift=-2*\smallspace, highlight on=<4>, highlight on=<8>] {$b$};
				\node[left=\labelspacing of x2l] {$A_2$};

				\node[right=0.4*\bigspace of x1, yshift=-2*\smallspace, align=center] {\LARGE$\oplus$};

				\node[c] (x3rl) [below=\smallspace of x2r, xshift=-2*\smallspace, highlight on=<10>, highlight on=<11>, highlight on=<12>] {$c$};
				\node[right=\labelspacing of x3rl] {$A_3$};

				\node[r, below=\smallspace of x3rl, xshift=-2*\smallspace, highlight on=<5>, highlight on=<6>] (zero) {$0$};
				\node[r, below=\smallspace of x3rl, highlight on=<7>] (one) {$1$};

				\draw[zeroarrow] (x1) -- (x2l);
				\draw[onearrow] (x1) -- (x3rl);

				\draw[zeroarrow] (x2l) -- (zero);
				\draw[onearrow] (x2l) -- (one);

				\draw[zeroarrow] (x3rl) -- (one);
				\draw[onearrow] (x3rl) -- (zero);

				% g
				\node[c, right=\bigspace of x1, highlight on=<3>, highlight on=<9>] (y1) {$a$};
				\node[right=\labelspacing of y1] {$B_1$};

				\node[c] (y2r) [below=\smallspace of y1, xshift=2*\smallspace, highlight on=<10>] {$b$};
				\node[right=\labelspacing of y2r] {$B_2$};

				\node[c] (y3lr) [below=\smallspace of y2r, xshift=-2*\smallspace, highlight on=<4>, highlight on=<5>,  highlight on=<6>, highlight on=<7>, highlight on=<8>] {$c$};
				\node[left=\labelspacing of y3lr] {$B_3$};

				\node[r, below=\smallspace of y3lr, highlight on=<11>] (yzero) {$0$};
				\node[r, below=\smallspace of y3lr, xshift=2*\smallspace, highlight on=<12>] (yone) {$1$};

				\draw[zeroarrow] (y1) -- (y3lr);
				\draw[onearrow] (y1) -- (y2r);

				\draw[zeroarrow] (y2r) -- (yzero);
				\draw[onearrow] (y2r) -- (yone);

				\draw[zeroarrow] (y3lr) -- (yone);
				\draw[onearrow] (y3lr) -- (yzero);

				% result

				\node[c, right=2*\bigspace of y1, visible on=<3->, highlight on=<3>, highlight on=<9>] (r1) {$a$};
				\node[left=\labelspacing of r1, visible on=<3->] {$A_1, B_1$};

				\node[c] (r2l) [below=\smallspace of r1, xshift=-2*\smallspace, visible on=<4->, highlight on=<4>, highlight on=<8>] {$b$};
				\node[left=\labelspacing of r2l, visible on=<4->] {$A_2, B_3$};
				\draw[zeroarrow, visible on=<4->] (r1) -- (r2l);

				\node[c] (r3ll) [below=\smallspace of r2l, xshift=-2*\smallspace, visible on=<5->, highlight on=<5>, highlight on=<11>] {$c$};
				\node[left=\labelspacing of r3ll, visible on=<5->] {$B_3$};
				\draw[zeroarrow, visible on=<5->] (r2l) -- (r3ll);

				\node[r, below=\smallspace of r3ll, visible on=<6->, highlight on=<6>] (rzero) {$0$};
				\node[r, below=\smallspace of r3ll, xshift=2*\smallspace, visible on=<6->, highlight on=<6>] (rone) {$1$};
				\draw[zeroarrow, visible on=<6->] (r3ll) -- (rone);
				\draw[onearrow, visible on=<6->] (r3ll) -- (rzero);

				\node[c] (r3lr) [below=\smallspace of r2l, visible on=<7->, highlight on=<7>, highlight on=<12>] {$c$};
				\node[right=\labelspacing of r3lr, visible on=<7->] {$\neg B_3$};
				\draw[onearrow, visible on=<7->] (r2l) -- (r3lr);
				\draw[zeroarrow, visible on=<7->] (r3lr) -- (rzero);
				\draw[onearrow, visible on=<7->] (r3lr) -- (rone);

				\node[c] (r2r) [below=\smallspace of r1, xshift=0, visible on=<10->, highlight on=<10>] {$b$};
				\node[below=\labelspacing of r2r, visible on=<10->] {$A_3,B_2$};
				\draw[onearrow, visible on=<10->] (r1) -- (r2r);

				\draw[zeroarrow, visible on=<11->] (r2r) -- (r3ll);
				\draw[onearrow, visible on=<12->] (r2r) -- (r3lr);
			\end{tikzpicture}
			\vfill{}
			\only<1-2>{\phantom{label}}
			\only<3>{Top-Variable $a$}
			\only<4>{Top-Variable $b$, Abstieg ins Low-Kind}
			\only<5>{Top-Variable $c$, Abstieg ins Low-Kind}
			\only<6>{$0 \oplus B_3 \equiv B_3$}
			\only<7>{$1 \oplus B_3 \equiv \neg B_3$}
			\only<8>{Top-Variable $b$}
			\only<9>{Top-Variable $a$}
			\only<10>{Top-Variable $b$, Abstieg ins High-Kind}
			\only<11>{$A_3 \oplus 0 \equiv A_3$}
			\only<12>{$A_3 \oplus 1 \equiv \neg A_3$}
		\end{center}
	}
	\only<13->{
		\begin{center}
			\begin{tikzpicture}

				\def\smallspace{0.75cm}
				\def\bigspace{3cm}
				\def\margin{0.75cm}
				\def\labelspacing{0cm}

				% f
				\node[c] (x1) {$a$};
				\node[right=\labelspacing of x1] {$A_1$};

				\node[c] (x2l) [below=\smallspace of x1, xshift=-2*\smallspace] {$b$};
				\node[left=\labelspacing of x2l] {$A_2$};

				\node[right=0.4*\bigspace of x1, yshift=-2*\smallspace, align=center] {\LARGE$\oplus$};

				\node[c] (x3rl) [below=\smallspace of x2r, xshift=-2*\smallspace] {$c$};
				\node[right=\labelspacing of x3rl] {$A_3$};

				\node[r, below=\smallspace of x3rl, xshift=-2*\smallspace] (zero) {$0$};
				\node[r, below=\smallspace of x3rl] (one) {$1$};

				\draw[zeroarrow] (x1) -- (x2l);
				\draw[onearrow] (x1) -- (x3rl);

				\draw[zeroarrow] (x2l) -- (zero);
				\draw[onearrow] (x2l) -- (one);

				\draw[zeroarrow] (x3rl) -- (one);
				\draw[onearrow] (x3rl) -- (zero);

				% g
				\node[c, right=\bigspace of x1] (y1) {$a$};
				\node[right=\labelspacing of y1] {$B_1$};

				\node[c] (y2r) [below=\smallspace of y1, xshift=2*\smallspace] {$b$};
				\node[right=\labelspacing of y2r] {$B_2$};

				\node[c] (y3lr) [below=\smallspace of y2r, xshift=-2*\smallspace] {$c$};
				\node[left=\labelspacing of y3lr] {$B_3$};

				\node[r, below=\smallspace of y3lr] (yzero) {$0$};
				\node[r, below=\smallspace of y3lr, xshift=2*\smallspace] (yone) {$1$};

				\draw[zeroarrow] (y1) -- (y3lr);
				\draw[onearrow] (y1) -- (y2r);

				\draw[zeroarrow] (y2r) -- (yzero);
				\draw[onearrow] (y2r) -- (yone);

				\draw[zeroarrow] (y3lr) -- (yone);
				\draw[onearrow] (y3lr) -- (yzero);

				% result	
				\node[c, right=2*\bigspace-\smallspace of y1, visible on=<13>] (r1) {$a$};

				\node[c] (r2l) [below=\smallspace of r1] {$b$};
				\draw[zeroarrow, visible on=<13>] (r1) to[bend right] (r2l);

				\node[c] (r3ll) [below=\smallspace of r2l, xshift=-\smallspace] {$c$};
				\draw[zeroarrow] (r2l) -- (r3ll);

				\node[r, below=\smallspace of r3ll] (rzero) {$0$};
				\node[r, below=\smallspace of r3ll, xshift=2*\smallspace] (rone) {$1$};
				\draw[zeroarrow] (r3ll) -- (rone);
				\draw[onearrow] (r3ll) -- (rzero);

				\node[c] (r3lr) [below=\smallspace of r2l,xshift=\smallspace] {$c$};
				\draw[onearrow] (r2l) -- (r3lr);
				\draw[zeroarrow] (r3lr) -- (rzero);
				\draw[onearrow] (r3lr) -- (rone);

				\draw[onearrow, visible on=<13>] (r1) to[bend left] (r2l);

			\end{tikzpicture}
			\vfill{}
			\only<13>{I-Reduktion}
			\only<14>{S-Reduktion}
		\end{center}
	}
\end{frame}

\begin{frame}[c]{}{}
	\begin{center}
		\LARGE Fragen?
	\end{center}
\end{frame}

\begin{frame}[c, fragile]{Links}{}
	\begin{itemize}
		\item Zulip: \href{https://zulip.in.tum.de/#narrow/channel/3255-ERA-Tutorium-.E2.80.93-Mi-1600-3}{\enquote{ERA Tutorium -- Mi-1600-3}}
		      bzw. \href{https://zulip.in.tum.de/#narrow/channel/3264-ERA-Tutorium-.E2.80.93-Fr-1500-1}{\enquote{ERA Tutorium -- Fr-1500-1}}
		\item \href{https://www.moodle.tum.de/course/view.php?id=111440}{ERA-Moodle-Kurs}
		\item \href{https://artemis.tum.de/courses/516}{ERA-Artemis-Kurs}
		\item \href{https://en.wikipedia.org/wiki/Binary_decision_diagram}{Wikipedia zu BDDs}
	\end{itemize}
\end{frame}

\maketitle

\end{document}
