%%
% This is an Overleaf template for presentations
% using the TUM Corporate Desing https://www.tum.de/cd
%
% For further details on how to use the template, take a look at our
% GitLab repository and browse through our test documents
% https://gitlab.lrz.de/latex4ei/tum-templates.
%
% The tumbeamer class is based on the beamer class.
% If you need further customization please consult the beamer class guide
% https://ctan.org/pkg/beamer.
% Additional class options are passed down to the base class.
%
% If you encounter any bugs or undesired behaviour, please raise an issue
% in our GitLab repository
% https://gitlab.lrz.de/latex4ei/tum-templates/issues
% and provide a description and minimal working example of your problem.
%%

\PassOptionsToClass{onlytextwidth}{beamer}

\documentclass[
  german,            % define the document language (english, german)
  aspectratio=169,    % define the aspect ratio (169, 43)
  % handout=2on1,       % create handout with multiple slides (2on1, 4on1)
  % partpage=false,     % insert page at beginning of parts (true, false)
  % sectionpage=true,   % insert page at beginning of sections (true, false)
]{tumbeamer}


% load additional packages
\usepackage{tikz}
\usepackage{circuitikz}
\usepackage{url}
\usepackage{hyperref}
\usepackage{pgf}
\usepackage{pgfplots}
\usepackage{babel}[ngerman]
\usepackage{csquotes}[autostyle]
\usepackage[useregional]{datetime2}
\usepackage{float}
\usepackage{graphicx}
\usepackage{amsmath}
\usepackage{amssymb}
\usepackage{xcolor}
\usepackage[cache=true]{minted}
\usemintedstyle{borland}
\usepackage{listings}
\usepackage{tikz-timing}

% tikz  
\usetikzlibrary{fit, matrix, calc, arrows, arrows.meta, positioning, patterns, patterns.meta, overlay-beamer-styles, riscvproc, automata}
\def\TikZ{Ti\emph{k}Z}
\def\Circuitikz{Circui\TikZ}

% processor
\ctikzset{processor/font=\fontsize{10}{12}\fontfamily{lmss}\selectfont}
\tikzset{font=\fontsize{10}{12}\fontfamily{lmss}\selectfont}
\definecolor{modblue}{HTML}{288DCC}

% https://tex.stackexchange.com/a/7045
\newcommand*\circled[1]{\tikz[baseline=(char.base)]{
		\node[shape=circle,draw,inner sep=2pt, font=\scriptsize] (char) {#1};}}

\newcommand*\colorcirc[1]{\tikz[baseline=(char.base)]{
		\node[shape=circle,fill=#1,inner sep=2pt] {};}}

\newcommand\n[1]{\mkern1mu\overline{\mkern-1mu#1}}

% requires circuitikz >= 1.1.0
% for distros with older distributions, install TeX Live manually
% instead of using your package manager
% see: https://tug.org/texlive/quickinstall.html
\ctikzset{logic ports=ieee}

% minted
\setminted{
	fontsize=\small, 
	frame=none,
	breaklines=true,
}

\captionsetup{labelformat=empty}

% image path
\graphicspath{ {./resources/}, {../riscv-circuitikz/figures/} }

% beamer
\setbeamercolor{footnote}{fg=black}
\setbeamercolor{footnote mark}{fg=black}
\renewcommand{\thempfootnote}{\arabic{mpfootnote}}


% presentation metadata
\title{Übung 09: Automaten \\und
	Multi-Cycle-Prozessor}
\subtitle{Einführung in die Rechnerarchitektur}
\author{Niklas Ladurner}

\institute{\theChairName\\\theDepartmentName\\\theUniversityName}
\date{\DTMdisplaydate{2025}{12}{12}{-1}}

\footline{\insertauthor~|~\insertshorttitle~|~\insertshortdate}


% macro to configure the style of the presentation
\TUMbeamersetup{
  title page = TUM tower,         % style of the title page
  part page = TUM toc,            % style of part pages
  section page = TUM toc,         % style of section pages
  content page = TUM more space,  % style of normal content pages
  tower scale = 1.0,              % scaling factor of TUM tower (if used)
  headline = TUM threeliner,      % which variation of headline to use
  footline = TUM default,         % which variation of footline to use
  % configure on which pages headlines and footlines should be printed
  headline on = {title page},
  footline on = {every page, title page=false},
}

\begin{document}

\maketitle

\begin{frame}[c, fragile]{}{}
	\begin{center}
		\LARGE  Keine Garantie für die Richtigkeit der Tutorfolien.

		\Large Bei Unklarheiten/Unstimmigkeiten haben VL/ZÜ-Folien recht!
	\end{center}
\end{frame}

\begin{frame}[fragile, c]{Endliche Automaten (1)}{}
	\begin{itemize}
		\item Repräsentiert Funktion einer sequentiellen (d.h. zustandsabhängigen) Schaltung
		\item Wechsel zwischen Zuständen; Folgezustand abhängig von aktuellem Zustand und Eingabe
		\item als Diagramm: Zustände $\rightarrow$ Kreise, Übergänge $\rightarrow$ Kanten, Bedingungen $\rightarrow$ Kantenbeschriftungen
		\item als 6-Tupel $(I, O, S, s_0, \delta, \lambda)$:
		      \begin{itemize}
			      \item $I$: Menge möglicher Eingaben, $O$: Menge möglicher Ausgaben
			      \item $S$: Zustandsmenge, $s_0\in S$: Startzustand
			      \item $\delta:  S\times I\rightarrow S$: Zustandsübergangsfunktion
			      \item $\lambda: S\rightarrow O$ (Moore), $\lambda: S\times I\rightarrow O$ (Mealy): Ausgabefunktion
		      \end{itemize}
	\end{itemize}
\end{frame}

\begin{frame}[fragile, c]{Endliche Automaten (2)}{}
	\begin{columns}[c]
		\begin{column}{0.5\textwidth}
			\begin{center}
				\textbf{Moore-Automat}\\
				{\scriptsize Ausgabe abhängig von aktuellem Zustand}\\
				\resizebox{!}{0.65\textheight}{
					\begin{tikzpicture}[initial text=]
						\node[state with output, initial] (q0) {$q_0$  \nodepart{lower} $00$};
						\node[state with output] (q1) [above right=of q0] {$q_1$  \nodepart{lower} $01$};
						\node[state with output] (q2) [below right=of q0] {$q_2$  \nodepart{lower} $10$};
						\node[state with output] (q3)  [below right=of q1] {$q_3$  \nodepart{lower} $11$};

						\path[->, every node/.style={execute at begin node=$, execute at end node=$}]
						(q0) edge node [above left]  {0} (q1)
						edge node [below left]  {1} (q2)
						(q1) edge node [above right] {1} (q3)
						edge [loop above] node {0} ()
						(q2) edge node [below right] {0} (q3)
						edge [loop below] node {1} ()
						(q3) edge [loop right] node {0} ()
						(q3) edge node [above] {1} (q0);
					\end{tikzpicture}
				}
			\end{center}
		\end{column}
		\begin{column}{0.5\textwidth}
			\begin{center}
				\textbf{Mealy-Automat}\\
				{\scriptsize Ausgabe abhängig von aktuellem Zustand + Eingabe}\\
				\resizebox{!}{0.65\textheight}{
					\begin{tikzpicture}[initial text=]
						\node[state, initial] (q0) {$q_0$  \nodepart{lower} $00$};
						\node[state] (q1) [above right=of q0] {$q_1$};
						\node[state] (q2) [below right=of q0] {$q_2$};
						\node[state] (q3)  [below right=of q1] {$q_3$};

						\path[->, every node/.style={execute at begin node=$, execute at end node=$}]
						(q0) edge node [above left]  {0/00} (q1)
						edge node [below left]  {1/10} (q2)
						(q1) edge node [above right] {1/11} (q3)
						edge [loop above] node {0/01} ()
						(q2) edge node [below right] {0/01} (q3)
						edge [loop below] node {1/10} ()
						(q3) edge [loop right] node {0/11} ()
						(q3) edge node [above] {1/00} (q0);
					\end{tikzpicture}
				}
			\end{center}
		\end{column}
	\end{columns}
	\begin{center}
		{\scriptsize $I=\{0, 1\}$, $O=\{00, 01, 10, 11\}$, $S=\{q_0, q_1, q_2, q_3\}, \delta, \lambda$ (abh. vom Typen)}
	\end{center}
\end{frame}

\begin{frame}[fragile, c]{Endliche Automaten: Realisierung}{}
	\begin{columns}[c]
		\begin{column}{0.5\textwidth}
			\begin{itemize}
				\item One-Hot-Kodierung: Genau 1 FF ist auf 1 (aktueller Zustand), einfach aber verschwenderisch
				\item Binärkodierung: FFs zusammen bilden Binärzahl des aktuellen Zustands, spart FFs aber komplexer
				\item Mikroprogrammiertes Steuerwerk: Nur ein Speicherbaustein, enthält vollständigen Automaten. Eingaben werden als Adressen interpretiert, sehr flexibel.
			\end{itemize}
		\end{column}
		\begin{column}{0.5\textwidth}
			\begin{table}[]
				\begin{tabular}{c|c|c}
					Zustand & One-Hot & Binär \\ \hline
					$S_0$   & $0001$              & $00$             \\
					$S_1$   & $0010$              & $01$             \\
					$S_2$   & $0100$              & $10$             \\
					$S_3$   & $1000$              & $11$
				\end{tabular}
			\end{table}
		\end{column}
	\end{columns}
\end{frame}


\begin{frame}[c, fragile]{Multi-Cycle-Prozessor}{}
	\begin{columns}[c]
		\begin{column}{0.6\textwidth}
			\begin{itemize}
				\item Grundidee: Aufteilung einer Instruktion in mehrere Schritte (d.h. mehrere Taktzyklen)
				\begin{itemize}
					\item Wiederverwendung von Hardware (bspw. ALU)
					\item kürzere kritische Pfade $\rightarrow$ höhere Taktfrequenz möglich\footnote[frame]{allerdings benötigt eine Instruktion auch mehrere Taktzyklen}
					\item komplexeres Steuerwerk (Zustandsautomat)
				\end{itemize}
				\item MC-Prozessoren heute nicht mehr relevant, aber Grundidee wird im pipelined Prozessor angewandt
			\end{itemize}
		\end{column}%
	\end{columns}	
\end{frame}

\begin{frame}[c]{Multi-Cycle-Prozessor: Schaltbild}{}
	\begin{center}
\resizebox{0.75\textwidth}{!}{
	\input{riscv_mc.tikz}
}
	\end{center}
\end{frame}

\begin{frame}[c]{Multi-Cycle-Prozessor: Automat}{}
	\begin{center}
\resizebox{0.6\textwidth}{!}{
	\input{riscv_mc_automaton.tikz}
}
	\end{center}
\end{frame}

\begin{frame}[c, fragile]{}{}
	\begin{center}
		\LARGE Fragen?
	\end{center}
\end{frame}

\begin{frame}[c, fragile]{Links}{}
	\begin{itemize}
	\item Zulip: \href{https://zulip.in.tum.de/#narrow/channel/3255-ERA-Tutorium-.E2.80.93-Mi-1600-3}{\enquote{ERA Tutorium -- Mi-1600-3}}
	bzw. \href{https://zulip.in.tum.de/#narrow/channel/3264-ERA-Tutorium-.E2.80.93-Fr-1500-1}{\enquote{ERA Tutorium -- Fr-1500-1}}
	\item \href{https://www.moodle.tum.de/course/view.php?id=111440}{ERA-Moodle-Kurs}
	\item \href{https://artemis.tum.de/courses/516}{ERA-Artemis-Kurs}
		\item \href{https://courses.edx.org/assets/courseware/v1/f06a2dc0c856f60ec0711e9f5e1c98cf/asset-v1:HarveyMuddX+ENGR85B+1T2023+type@asset+block/FinalReferences.pdf}{Prozessor-Assets (kein offizielles Material!)}
		\item \href{https://github.com/ladnik/riscv-circuitikz}{Meine \Circuitikz{}-library für schöne Prozessoren :)}
	\end{itemize}
\end{frame}

\maketitle

\end{document}
