%%
% This is an Overleaf template for presentations
% using the TUM Corporate Desing https://www.tum.de/cd
%
% For further details on how to use the template, take a look at our
% GitLab repository and browse through our test documents
% https://gitlab.lrz.de/latex4ei/tum-templates.
%
% The tumbeamer class is based on the beamer class.
% If you need further customization please consult the beamer class guide
% https://ctan.org/pkg/beamer.
% Additional class options are passed down to the base class.
%
% If you encounter any bugs or undesired behaviour, please raise an issue
% in our GitLab repository
% https://gitlab.lrz.de/latex4ei/tum-templates/issues
% and provide a description and minimal working example of your problem.
%%

\PassOptionsToClass{onlytextwidth}{beamer}

\documentclass[
  german,            % define the document language (english, german)
  aspectratio=169,    % define the aspect ratio (169, 43)
  % handout=2on1,       % create handout with multiple slides (2on1, 4on1)
  % partpage=false,     % insert page at beginning of parts (true, false)
  % sectionpage=true,   % insert page at beginning of sections (true, false)
]{tumbeamer}


% load additional packages
\usepackage{tikz}
\usepackage{circuitikz}
\usepackage{url}
\usepackage{hyperref}
\usepackage{pgf}
\usepackage{pgfplots}
\usepackage{babel}[ngerman]
\usepackage{csquotes}[autostyle]
\usepackage[useregional]{datetime2}
\usepackage{float}
\usepackage{graphicx}
\usepackage{amsmath}
\usepackage{amssymb}
\usepackage{xcolor}
\usepackage[cache=true]{minted}
\usemintedstyle{borland}
\usepackage{listings}
\usepackage{lstautogobble}
\usepackage{tikz-timing}
\usepackage{ifthen}
\usepackage{ulem}
\usepackage{enumerate}
\usepackage{minted}
\usepackage[dvipsnames]{xcolor}

% tikz  
\usetikzlibrary{fit, matrix, calc, arrows, arrows.meta, positioning, patterns, patterns.meta, overlay-beamer-styles, riscvproc, automata}
\def\TikZ{Ti\emph{k}Z}
\def\Circuitikz{Circui\TikZ}

% processor
\ctikzset{processor/font=\fontsize{10}{12}\fontfamily{lmss}\selectfont}
\tikzset{font=\fontsize{10}{12}\fontfamily{lmss}\selectfont}
\definecolor{modblue}{HTML}{288DCC}

% https://tex.stackexchange.com/a/7045
\newcommand*\circled[1]{\tikz[baseline=(char.base)]{
		\node[shape=circle,draw,inner sep=2pt, font=\scriptsize] (char) {#1};}}

\newcommand*\colorcirc[1]{\tikz[baseline=(char.base)]{
		\node[shape=circle,fill=#1,inner sep=2pt] {};}}

\newcommand\n[1]{\mkern1mu\overline{\mkern-1mu#1}}

% requires circuitikz >= 1.1.0
% for distros with older distributions, install TeX Live manually
% instead of using your package manager
% see: https://tug.org/texlive/quickinstall.html
\ctikzset{logic ports=ieee}

% minted
\setminted{
	fontsize=\small, 
	frame=none,
	breaklines=true,
}

\captionsetup{labelformat=empty}

% image path
\graphicspath{ {./resources/}, {../riscv-circuitikz/figures/} }

% beamer
\setbeamercolor{footnote}{fg=black}
\setbeamercolor{footnote mark}{fg=black}
\renewcommand{\thempfootnote}{\arabic{mpfootnote}}

% presentation metadata
\title{Übung 12: Pipelining}
\subtitle{Einführung in die Rechnerarchitektur}
\author{Niklas Ladurner}

\institute{\theChairName\\\theDepartmentName\\\theUniversityName}
\date{\DTMdisplaydate{2026}{01}{16}{-1}}

\footline{\insertauthor~|~\insertshorttitle~|~\insertshortdate}


% macro to configure the style of the presentation
\TUMbeamersetup{
  title page = TUM tower,         % style of the title page
  part page = TUM toc,            % style of part pages
  section page = TUM toc,         % style of section pages
  content page = TUM more space,  % style of normal content pages
  tower scale = 1.0,              % scaling factor of TUM tower (if used)
  headline = TUM threeliner,      % which variation of headline to use
  footline = TUM default,         % which variation of footline to use
  % configure on which pages headlines and footlines should be printed
  headline on = {title page},
  footline on = {every page, title page=false},
}

\setbeamercolor{enumerate item}{fg=black}

\begin{document}

\maketitle

\begin{frame}[c, fragile]{}{}
	\begin{center}
		\LARGE  Keine Garantie für die Richtigkeit der Tutorfolien.

		\Large Bei Unklarheiten/Unstimmigkeiten haben VL/ZÜ-Folien recht!
	\end{center}
\end{frame}

\def\cellwt{1.1cm}
\def\cellht{0.7cm}
\def\cellspc{1.25}
\tikzset{cell/.style={draw, minimum width=\cellwt, minimum height=\cellht, align=center, rounded corners=0.1cm},
	label/.style={draw=none, minimum width=\cellwt, minimum height=\cellht,text width=\cellwt, text depth=0pt, align=center},
	A/.style={fill=red!70},
	B/.style={fill=orange!70!},
	C/.style={fill=yellow!70},
	D/.style={fill=RoyalBlue!60},
	E/.style={fill=Orchid!70},
	F/.style={fill=ForestGreen!70},
	G/.style={fill=TealBlue!70},
	empty/.style={fill=gray!15}}

\begin{frame}[fragile, c]{Pipelining}{}
	\begin{columns}
		\begin{column}[c]{0.45\textwidth}
			\begin{itemize}
				\item Parallele Verarbeitung von mehreren Instruktionen
				\item Aufteilung in 5 Teilschritte: Fetch, Decode, Execute, Memory, Writeback
				\item Daten- und Kontrollpfad des Prozessors wird aufgetrennt: Register zur Zwischenspeicherung
				\item maximaler Speedup: Anzahl $n$ der Pipelinestufen
			\end{itemize}.
		\end{column}
		\begin{column}[c]{0.5\textwidth}
			\resizebox{\textwidth}{!}{
				\begin{tikzpicture}
					% row labels
					\node[rotate=90] at (0,2) {Zeitschritt};
					\foreach \r in {1,...,5}
					\node[label] at (0.4,5-\r) {\r};

					% column labels
					\foreach \c/\name in {1/Fetch,2/Decode,3/Execute,4/Memory,5/Writeb.}
					\node[label] at (\cellspc*\c,4.75) {\name};

					% grid
					\foreach \r in {1,...,5}
					\foreach \c in {1,...,5}
					\node[cell,empty] at (\cellspc*\c,5-\r) {};

					% i1
					\foreach \r/\c in {1/1,2/2,3/3,4/4,5/5}
					\node[cell,A] at (\cellspc*\c,5-\r) {\texttt{i1}};

					% i2
					\foreach \r/\c in {2/1,3/2,4/3,5/4}
					\node[cell,B] at (\cellspc*\c,5-\r) {\texttt{i2}};

					% i3
					\foreach \r/\c in {3/1,4/2,5/3}
					\node[cell,C] at (\cellspc*\c,5-\r) {\texttt{i3}};

					% i4
					\foreach \r/\c in {4/1,5/2}
					\node[cell,D] at (\cellspc*\c,5-\r) {\texttt{i4}};

					% i5
					\foreach \r/\c in {5/1}
					\node[cell,E] at (\cellspc*\c,5-\r) {\texttt{i5}};

				\end{tikzpicture}}
		\end{column}
	\end{columns}
\end{frame}

\begin{frame}[fragile, c]{Pipelined Prozessor: Schaltbild}{}
	\begin{center}
		\resizebox{0.8\textwidth}{!}{
			\input{riscv_pl.tikz}
		}
	\end{center}
\end{frame}

\begin{frame}[fragile, c]{Datenabhängigkeiten}{}
	\begin{columns}
		\begin{column}[c]{0.5\textwidth}
			\begin{itemize}
				\item<1-> \textbf{RAW} (Read-After-Write): potentiell ein Problem, falls schreibende Instruktion noch nicht zurückgeschrieben hat
				\item<2-> \textbf{WAR} (Write-After-Read): Reihenfolge der betroffenen Instruktionen darf nicht vertauscht werden
				\item<3-> \textbf{WAW} (Write-After-Write): Reihenfolge der betroffenen Instruktionen darf nicht vertauscht werden
			\end{itemize}
		\end{column}%
		\begin{column}[c]{0.5\textwidth}
			\begin{listing}[H]
				\centering
				\begin{minipage}{\widthof{\texttt{add t2, t0, s5}}}
					\begin{minted}[escapeinside=||, autogobble=true, frame=none]{text}
				lw |\color{red}{|t0|}|, 0(a1)
				add t2, |\color{red}{|t0|}|, s5
			\end{minted}
				\end{minipage}
				\caption{RAW-Abhängigkeit bzgl. \texttt{t0}}
			\end{listing}
			\pause
			\begin{listing}[H]
				\centering
				\begin{minipage}{\widthof{\texttt{add t2, t0, s5}}}
					\begin{minted}[escapeinside=||, autogobble=true, frame=none]{text}
						xor a0, |\color{blue}{|a2|}|, a1
						sub |\color{blue}{|a2|}|, a4, a5
					\end{minted}
				\end{minipage}
				\caption{WAR-Abhängigkeit bzgl. \texttt{a2}}
			\end{listing}
			\pause
			\begin{listing}[H]
				\centering
				\begin{minipage}{\widthof{\texttt{add t2, t0, s5}}}
					\begin{minted}[escapeinside=||, autogobble=true, frame=none]{text}
					and |\color{green}{|s0|}|, s1, s2
					lw |\color{green}{|s0|}|, 0(s3)
				\end{minted}
				\end{minipage}
				\caption{WAW-Abhängigkeit bzgl. \texttt{s0}}
			\end{listing}
		\end{column}
	\end{columns}
\end{frame}

\begin{frame}[c, fragile]{Pipelinekonflikte}{}
	\vspace{-0.5cm}
	\begin{columns}
		\begin{column}{0.45\textwidth}
			\begin{center}\textbf{Data Hazards}\\(Datenkonflikte\footnote[frame]{Nicht zu verwechseln mit den vorher genannten Daten\textit{abhängigkeiten} (RAW, WAR, WAW)})\end{center}
			\begin{itemize}
				\item können nur bei RAW auftreten\footnote[frame]{müssen aber nicht!}
				\item abhängige Instruktion in Decode, aber Ergebnis noch nicht zurückgeschrieben
			\end{itemize}
		\end{column}
		\begin{column}{0.45\textwidth}
			\begin{center}\textbf{Control Hazards}\\(Steuerkonflikte)\end{center}
			\begin{itemize}
				\item Kontrollflussänderungen durch branches/jumps
				\item falsche Sprungentscheidung: falsche Instruktionen in Pipeline geladen
			\end{itemize}
		\end{column}
	\end{columns}
\end{frame}

\begin{frame}[c, fragile]{Lösung von Konflikten: Data Hazards}
	\begin{columns}
		\begin{column}[c]{0.5\textwidth}
			Mindestens \textbf{3 Befehle} zwischen zwei Instruktionen mit RAW-Abhängigkeit:
			\begin{itemize}
				\item<2-> NOPs (Stalling)
				\item<3-> Befehlsumordnung (ohne Änderung der Semantik)
			\end{itemize}
			\vspace*{\baselineskip}
			\visible<4->{Alternativ:}
			\begin{itemize}
				\item<4-> Forwarding: noch nicht zurückgeschriebenes Ergebnis kann von einer Stage in eine andere geleitet werden
			\end{itemize}
		\end{column}
		\begin{column}[c]{0.5\textwidth}
			\centering

			\resizebox{\textwidth}{!}{
				\begin{tikzpicture}
					% row labels
					\node[rotate=90] at (0,2) {Zeitschritt};
					\foreach \r in {1,...,5}
					\node[label] at (0.4,5-\r) {\r};

					% column labels
					\foreach \c/\name in {1/Fetch,2/Decode,3/Execute,4/Memory,5/Writeb.}
					\node[label] at (\cellspc*\c,4.75) {\name};

					% grid
					\foreach \r in {1,...,5}
					\foreach \c in {1,...,5}
					\node[cell,empty] at (\cellspc*\c,5-\r) {};

					% i1
					\foreach \r/\c in {1/1,2/2,3/3,4/4,5/5}
					\node[cell,A] at (\cellspc*\c,5-\r) {\texttt{i1}};

					% conflict and forwarding
					\only<1, 4>{
						% i2
						\foreach \r/\c in {2/1,3/2,4/3,5/4}
						\node[cell,B] at (\cellspc*\c,5-\r) {\texttt{i2}};

						% i3
						\foreach \r/\c in {3/1,4/2,5/3}
						\node[cell,C] at (\cellspc*\c,5-\r) {\texttt{i3}};

						% i4
						\foreach \r/\c in {4/1,5/2}
						\node[cell,D] at (\cellspc*\c,5-\r) {\texttt{i4}};

						% i5
						\foreach \r/\c in {5/1}
						\node[cell,E] at (\cellspc*\c,5-\r) {\texttt{i5}};
					}
				
				% stalling
				\only<2>{
					% nop
					\foreach \r/\c in {2/1,3/2,4/3,5/4}
					\node[cell,empty] at (\cellspc*\c,5-\r) {\texttt{NOP}};
					
					% nop
					\foreach \r/\c in {3/1,4/2,5/3}
					\node[cell,empty] at (\cellspc*\c,5-\r) {\texttt{NOP}};
					
					% nop
					\foreach \r/\c in {4/1,5/2}
					\node[cell,empty] at (\cellspc*\c,5-\r) {\texttt{NOP}};
					
					% i2
					\foreach \r/\c in {5/1}
					\node[cell,B] at (\cellspc*\c,5-\r) {\texttt{i2}};
				}
			
			% reordering
			\only<3>{	
				% i3
				\foreach \r/\c in {2/1,3/2,4/3,5/4}
				\node[cell,C] at (\cellspc*\c,5-\r) {\texttt{i3}};
				
				% i4
				\foreach \r/\c in {3/1,4/2,5/3}
				\node[cell,D] at (\cellspc*\c,5-\r) {\texttt{i4}};
				
				% i5
				\foreach \r/\c in {4/1,5/2}
				\node[cell,E] at (\cellspc*\c,5-\r) {\texttt{i5}};
				
				% i2
				\foreach \r/\c in {5/1}
				\node[cell,B] at (\cellspc*\c,5-\r) {\texttt{i2}};
			}
		
			\visible<4>{
				\draw[very thick, ->] (\cellspc*4,\cellht - 2pt) -- ++(0, -0.3*\cellht) -| (\cellspc*3,\cellht - 2pt);
			}
		
				\end{tikzpicture}
			}

			\begin{center}
				Konflikt zwischen \texttt{i1} und \texttt{i2}
			\end{center}
		\end{column}
	\end{columns}
\end{frame}

\begin{frame}[c, fragile]{Lösung von Konflikten: Control  Hazards}
	\begin{columns}
		\begin{column}[c]{0.5\textwidth}
			Mindestens \textbf{2 Befehle} zwischen Sprungentscheidung und
			möglicherweise falsch geladener Instruktion:
			\begin{itemize}
				\item<2-> NOPs (Stalling)
			\end{itemize}
			\vspace*{\baselineskip}
			\visible<3->{Alternativ:}
			\begin{itemize}
				\item<3-> Flushing: Falls Vorhersage der branch prediction falsch war, müssen geladene Instruktionen aus Pipeline entfernt werden
			\end{itemize}
		\end{column}
	\begin{column}[c]{0.5\textwidth}
		\centering
		
		\resizebox{\textwidth}{!}{
			\begin{tikzpicture}
				% row labels
				\node[rotate=90] at (0,2) {Zeitschritt};
				\foreach \r in {1,...,5}
				\node[label] at (0.4,5-\r) {\r};
				
				% column labels
				\foreach \c/\name in {1/Fetch,2/Decode,3/Execute,4/Memory,5/Writeb.}
				\node[label] at (\cellspc*\c,4.75) {\name};
				
				% grid
				\foreach \r in {1,...,5}
				\foreach \c in {1,...,5}
				\node[cell,empty] at (\cellspc*\c,5-\r) {};
				
				% i1
				\foreach \r/\c in {1/1,2/2,3/3,4/4,5/5}
				\node[cell,A] at (\cellspc*\c,5-\r) {\texttt{i1}};
				
				% conflict
				\only<1>{
					% i2
					\foreach \r/\c in {2/1,3/2,4/3,5/4}
					\node[cell,B] at (\cellspc*\c,5-\r) {\texttt{i2}};
					
					% i3
					\foreach \r/\c in {3/1,4/2,5/3}
					\node[cell,C] at (\cellspc*\c,5-\r) {\texttt{i3}};
					
					% i4
					\foreach \r/\c in {4/1,5/2}
					\node[cell,D] at (\cellspc*\c,5-\r) {\texttt{i4}};
					
					% i5
					\foreach \r/\c in {5/1}
					\node[cell,E] at (\cellspc*\c,5-\r) {\texttt{i5}};
				}
				
				% stalling
				\only<2>{
					% nop
					\foreach \r/\c in {2/1,3/2,4/3,5/4}
					\node[cell,empty] at (\cellspc*\c,5-\r) {\texttt{NOP}};
					
					% nop
					\foreach \r/\c in {3/1,4/2,5/3}
					\node[cell,empty] at (\cellspc*\c,5-\r) {\texttt{NOP}};
					
					% i22
					\foreach \r/\c in {4/1,5/2}
					\node[cell,F] at (\cellspc*\c,5-\r) {\texttt{i22}};
					
					% i23
					\foreach \r/\c in {5/1}
					\node[cell,G] at (\cellspc*\c,5-\r) {\texttt{i23}};
				}
				
				% flushing
				\only<3>{
					% nop
					\foreach \r/\c in {4/3,5/4}
					\node[cell,empty] at (\cellspc*\c,5-\r) {\texttt{NOP}};
					
					% nop
					\foreach \r/\c in {4/2,5/3}
					\node[cell,empty] at (\cellspc*\c,5-\r) {\texttt{NOP}};
					
					% i2
					\foreach \r/\c in {2/1,3/2}
					\node[cell,B] at (\cellspc*\c,5-\r) {\texttt{i2}};
					
					% i3
					\foreach \r/\c in {3/1}
					\node[cell,C] at (\cellspc*\c,5-\r) {\texttt{i3}};
					
					% i22
					\foreach \r/\c in {4/1,5/2}
					\node[cell,F] at (\cellspc*\c,5-\r) {\texttt{i22}};
					
					% i23
					\foreach \r/\c in {5/1}
					\node[cell,G] at (\cellspc*\c,5-\r) {\texttt{i23}};
				}
			\end{tikzpicture}
		}
		
		\begin{center}
			Branch durch \texttt{i1}
		\end{center}
	\end{column}
	\end{columns}
\end{frame}

\begin{frame}[c, fragile]{Links}{}
	\begin{itemize}
		\item \href{https://zulip.in.tum.de/#narrow/channel/3255-ERA-Tutorium-.E2.80.93-Mi-1600-3}{\enquote{ERA Tutorium -- Mi-1600-3}}
		      bzw. \href{https://zulip.in.tum.de/#narrow/channel/3264-ERA-Tutorium-.E2.80.93-Fr-1500-1}{\enquote{ERA Tutorium -- Fr-1500-1}}
		\item \href{https://www.moodle.tum.de/course/view.php?id=111440}{ERA-Moodle-Kurs}
		\item \href{https://artemis.tum.de/courses/516}{ERA-Artemis-Kurs}
	\end{itemize}
\end{frame}

\maketitle

\end{document}
